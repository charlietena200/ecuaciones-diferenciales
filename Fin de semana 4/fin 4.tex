\documentclass[a4paper,10pt]{article}
\usepackage[left=2cm,top=2cm,right=2cm,bottom=2cm]{geometry}
\usepackage[utf8]{inputenc}
\usepackage{amsthm}
\usepackage{graphicx}
\graphicspath{ {images/} }


\newcommand{\mN}{{\mathbb N}}
\newcommand{\mZ}{{\mathbb Z}}
\newcommand{\cZ}{{\mathcal Z}}
\newcommand{\fL}{{\mathfrak L}}
\newcommand{\dst}{\displaystyle}

%opening
\title{}
\author{}
\date{}

\begin{document}

\maketitle
Carlos Gallegos\\\\
Actividad de fin de semana\\\\
1. xy' = $\sqrt{x^2-y^2} + y$ \\\\
Notamos que no es exacta, por lo que haremos cambio de variable tomando $u=\frac{y}{x} \rightarrow y=ux$ por lo que dy= xdu + udx\\\\
Primero dividimos todo entre x y nos queda:\\\
\centerline{y' = $\frac{\sqrt{x^2 - y^2}}{x} + \frac{y}{x} $}\\
Simplificando la primer división:\\
\centerline{$\sqrt{\frac{x^2}{x^2}-\frac{y^2}{x^2}} + \frac{y}{x}= \sqrt{1-(\frac{y}{x})^2} + \frac{y}{x}$}\\\\
Entonces hacemos el cambio de variable y nos queda:\\\\
\centerline{y'= $ \sqrt{1-u^2} + u$ }\\\\
Sustituyendo:\\\\
\centerline{$\frac{dy}{dx}=  \sqrt{1-u^2} + u \quad \rightarrow \quad \frac{xdu}{dx} + u = \sqrt{1-u^2} + u$}\\\\
\centerline{$\frac{xdu}{dx} = \sqrt{1-u^2} = x u'$}\\\\
Por lo que ahora si podemos "separar" nuestras variables de la siguiente forma:\\\\
\centerline{$\frac{u'}{\sqrt{1-u^2}}  = \frac{1}{x}$}\\\\
Por lo que resolviendo por variables separables nos queda:\\\\
\centerline{$\int \frac{1}{\sqrt{1-u^2}} du= \int \frac{1}{x} dx$}\\\\
Notamos que las integrales salen directas, y nos dan:\\\\
\centerline{arcsen(u)= ln(x) + c}\\\\
Con el cambio de variable $u=\frac{y}{x}$\\\\
$arcsen(\frac{y}{x})= ln(x) + c $\\\\
Si aplicamos el seno en ambos lados de la ecuación, y despejamos x, tenemos una solución explícita:\\\\
\centerline{y= x sen(ln(x)) + c}\\\\\\
2. x+y-2+(x-y-4)y' = 0\\\\
Primero tomamos M(x,y)=x+y-2 y N(x,y)=x-y-4\\\\
$M_y = 1 \quad y \quad N_x=1 $ por lo que es exacta, entonces integrando:\\\\
$\int M dx = \frac{x^2}{2} + xy - 2x + c$\\\\
$\int N dy= xy - \frac{y^2}{2} -4y + c2$\\\\
Por lo que formamos un $\omega$  tal que $\omega(x,y)=\frac{x^2}{2}- \frac{y^2}{2} +xy -2x -4y$.\\\\
Notamos que $\omega_xy = \omega_yx$.\\
Por lo que tenemos como solución implícita:\\\\
\centerline{$\frac{x^2}{2}- \frac{y^2}{2} +xy -2x -4y = c $}\\\\\
3. $\frac{2xy^3}{3} + (x^2y^2-1)y'=0$\\\\ 
Primero tomamos M(x,y)=$\frac{2xy^3}{3}$ y N(x,y)=$(x^2y^2-1)$\\\\
 $M_y = 2xy^2 \quad y \quad N_x=2xy^2$ por que lo que es exacta, integrando tenemos que:\\\\
 $\int M dx = \frac{y^3x^2}{3} + c$\\\\
 $\int N dy =\frac{x^2y^3}{3}-y +c2$\\\\
 Por lo que formamos un $\omega$  tal que $\omega(x,y)=\frac{y^3x^2}{3} -y$.\\\\
 Por lo que tenemos como solución implícita $\omega(x,y)=\frac{y^3x^2}{3} -y$ donde $\omega(x,y) = c$\\
\end{document}
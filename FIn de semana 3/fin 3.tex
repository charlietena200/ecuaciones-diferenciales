\documentclass[a4paper,10pt]{article}
\usepackage[left=2cm,top=2cm,right=2cm,bottom=2cm]{geometry}
\usepackage[utf8]{inputenc}
\usepackage{amsthm}
\usepackage{graphicx}
\graphicspath{ {images/} }


\newcommand{\mN}{{\mathbb N}}
\newcommand{\mZ}{{\mathbb Z}}
\newcommand{\cZ}{{\mathcal Z}}
\newcommand{\fL}{{\mathfrak L}}
\newcommand{\dst}{\displaystyle}

%opening
\title{}
\author{}
\date{}

\begin{document}

\maketitle
Carlos Gallegos\\\\
Actividad de fin de semana\\\\
1.$y' - y^3 = (x-y')y^2$\\\\
Nos damos cuenta que la ecuación no es separable porque no encotramos una forma de tener las x separadas de las y. Por ello, no la podemos resolver por el método de variables separables.\\\\\\
2.$y'= cos(x+y)+cos(x-y)$\\\\
Aplicamos las propiedades trigonométricas y nos queda que:\\\\
\centerline{= $ cosx cosy + senx seny + cosx cosy - senx seny$}\\\\
Simplifiacndo nos queda: $ = 2cosx seny $\\\\
Notamos que la podemos separar en g(x)= 2cosx y h(x) = cosy\\\\
Ahora resolvemos, sea f(x)= $\int \frac{1}{cosy} dy$ entonces :\\\\
f(x) = $\int 2cosx dx = 2senx + c $\\\\
Por lo que encontramos una solución implícita.\\\\\\
3. sen(x-y)dx+dy=sen(x+y)dx\\\\
Aplicando propiedades trigonométricas nos queda:\\\\
\centerline{$(senx cosy - seny cosx) dx + dy = (senx cosy + sey cosx )dx$ }\\\\
Por lo que nos queda dy= $(2seny cosx) dx$. Notamos que es separable con g(x) = 2senx y h(x) = cosy\\\\
Tomamos f(x) como $\int \frac{1}{senx} dy$ y tenemos que :\\\\
\centerline{$f(x) = \int 2cosx dx = 2senx + c$}\\\\
Por lo que encontramos una solución implícita.\\\\\\
4. $y^2 dx - xy dy = x^2 y dy$\\\\


\end{document}